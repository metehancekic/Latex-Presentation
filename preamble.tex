%!TEX root = main.tex

\documentclass[hyperref={unicode=true},aspectratio=1610]{beamer}
\usetheme{Madrid}
\beamertemplatenavigationsymbolsempty

\setbeamersize{text margin left=0.6in,text margin right=0.6in} 
\usepackage{caption}
\captionsetup{labelformat=empty}
\usepackage[absolute,overlay]{textpos}
\usepackage[super]{nth}
\usepackage{mathtools}
\usefonttheme[onlymath]{serif}
\usepackage{bm}
\usepackage{tabularx}
\usepackage{amsmath}
\usepackage{booktabs} % To thicken table lines
\usepackage{pdfcomment}
\setbeamertemplate{itemize items}[default]
\setbeamertemplate{enumerate items}[default]

\newcommand{\ra}[1]{\renewcommand{\arraystretch}{#1}}

% \usepackage{beamerthemeshadow}
\setbeamertemplate{navigation symbols}{}
\usetheme{Madrid}
\definecolor{LightCyan}{rgb}{0.7,1,1}
\definecolor{LightGrey}{rgb}{0.95,0.95,0.95}
% \definecolor{DarkBlue}{rgb}{0.1,0.1,0.7}
\definecolor{Yellow}{rgb}{1,1,0.6}
% \definecolor{DarkRed}{rgb}{0.7,0.1,0.1}
% \definecolor{DarkRed}{rgb}{0.51,0.36,0}
\definecolor{DarkRed}{rgb}{0.56,0.44,.12}
\definecolor{LightRed}{rgb}{0.9,0.5,0.5}
\definecolor{DarkGreen}{rgb}{0,0.5,0}
% \definecolor{DarkBlue}{rgb}{.38, 0, .55}
\definecolor{DarkBlue}{rgb}{.13, 0.27, .55}
\usecolortheme[named=DarkBlue]{structure}


% \setbeamertemplate{title page}[default][colsep=-4bp,rounded=true]
\setbeamertemplate{headline}{}
\setbeamertemplate{footline}[frame number]{}
\setbeamercovered{transparent}
\setbeamertemplate{itemize item}[triangle]
\setbeamertemplate{itemize subitem}[circle]
\setbeamertemplate{itemize subsubitem}{--}
\setbeamertemplate{enumerate items}[default]
\usefonttheme[onlymath]{serif}
\setbeamercolor{item}{fg=DarkBlue} % color of bullets
\setbeamercolor{subitem}{fg=DarkBlue}
\setbeamercovered{invisible}

\usepackage{cite}
% \usepackage{amsmath,amssymb,amsfonts}
% \usepackage{algorithmic}
\usepackage{graphicx}
\usepackage{textcomp}

\usepackage{url}            % simple URL typesetting
\usepackage{booktabs}       % professional-quality tables
\usepackage{amsfonts}       % blackboard math symbols
\usepackage{nicefrac}       % compact symbols for 1/2, etc.
\usepackage{microtype}      % microtypography
\usepackage[compatibility=false]{caption}
\usepackage{subcaption}
\usepackage{ifthen}
\usepackage{bbm}
\usepackage{amssymb}
\usepackage{algorithmicx}
% \usepackage[cmex10]{amsmath}
\usepackage{amsthm}
\usepackage{mathtools}
\usepackage{bm}
% \usepackage{enumitem}
\usepackage[electronic]{ifsym}
% \usepackage{caption}
\usepackage{physics}

\captionsetup[figure]{labelformat=empty}

\usepackage{makecell,pbox,ragged2e,hhline}
% \usepackage[table, x11names, svgnames]{xcolor}

\usepackage{color,diagbox,tabularx,multirow}
\usepackage{multicol}

\usepackage{colortbl}

\usepackage{tikz}
\usetikzlibrary{fit, shapes.geometric, shapes.misc, arrows.meta, positioning, decorations.markings, matrix, calc, decorations.pathreplacing}
\tikzset{>={Latex[width=1mm,length=1.2mm]}}
\usepackage{smartdiagram}
\usesmartdiagramlibrary{additions}
\usepackage{pgfplots}
\pgfplotsset{compat=newest, ticks=none}
\usepackage{varwidth}
\tikzset{every picture/.style={/utils/exec={\fontfamily{lmss}}}}
\tikzset{circ/.style = {circle, draw=black!100, fill=blue!25, thin, minimum height=3mm, minimum width=3mm, 
  inner sep=1.2mm},
  rect2/.style = {rectangle, draw=black!100, fill=cyan!18, thin, minimum height=7.5mm, minimum width=11mm, 
  inner sep=1.5mm},
  circ2/.style = {circle, draw=black!100, fill=red!35, thin, minimum size=7mm},
  outer/.style = {rounded corners=0.07cm, draw=DarkBlue, inner sep = 0mm, minimum height=6mm, ultra thick},
  outer2/.style = {rounded corners=0.2cm, draw=black!100, dashed, inner sep = 3mm},
  encode/.style = {trapezium, draw=black!100, fill=green!20, trapezium angle=75, shape border rotate=90, shift = {(-0.75, 0)}, minimum width=5mm, minimum height=11mm, align=center},
  decode/.style = {trapezium, draw=black!100, fill=green!20, trapezium angle=-75, shape border rotate=90, minimum width=5mm, minimum height=11mm, align=center},
  rect3/.style = {rectangle, draw=black!100, fill=cyan!1, thin, minimum height=7.5mm, minimum width=11mm, 
  inner sep=1.5mm},
  bluelines/.style={smooth, very thick, blue!40!gray},
  pre/.style={draw, rectangle, minimum height=6mm, minimum width=15mm, align=center,fill=gray!20, inner sep = 0mm},
  add/.style={draw, rectangle, minimum height=6mm, minimum width=22.5mm, align=center,fill=DarkBlue!20, inner sep = 0mm},
  msg/.style={draw, rectangle, minimum height=6mm, minimum width=35mm, align=center,fill=gray!20, inner sep = 0mm},
  par/.style={draw, rectangle, minimum height=6mm, minimum width=22.5mm, align=center,fill=gray!20, inner sep = 0mm},
  long/.style={short, text width=1.5cm},
  brace/.style={decorate, decoration={brace,amplitude=4pt,mirror,raise=1pt}},
  }

\usepackage{steinmetz}
\usepackage{xpatch}
\xpatchcmd{\phase}{#2}{\hspace{0.8pt}\vphantom{\scalebox{0.8}{\tiny{,}}}#2\hspace{1.4pt}}{}{}
\newcommand{\implyarrow}{\mathrel{\text{\raisebox{1.3ex}{\rotatebox[origin=c]{90}{\mathhexbox37F}}}}}
\makeatletter
\newdimen\@widthOfTo%
\newdimen\@widthOfImplies%
\settowidth{\@widthOfTo}{$\to$}%
\settowidth{\@widthOfImplies}{$\Longrightarrow$}%
\pgfmathsetmacro{\@scaleFactorImplies}{\@widthOfTo/\@widthOfImplies}%
\newcommand*{\ScaledImplies}{\mathrel{\raisebox{0.3ex}{\scalebox{\@scaleFactorImplies}{\ensuremath{\Longrightarrow}}}}}%
\makeatother


\newcolumntype{x}[1]{>{\centering\arraybackslash}p{#1}}

\newcommand{\greyrule}{\arrayrulecolor{black!70}\midrule\arrayrulecolor{black}}

\newcommand*\circled[1]{\tikz[baseline=(char.base)]{% <---- BEWARE
            \node[shape=circle,draw,inner sep=2pt] (char) {#1};}}

\newcommand<>{\uncovergraphics}[2][{}]{
    % Taken from: <https://tex.stackexchange.com/a/354033/95423>
    \begin{tikzpicture}
    \node[anchor=south west,inner sep=0] (B) at (4,0)
        {\includegraphics[#1]{#2}};
    \alt#3{}{%
        \fill [draw=none, fill=white, fill opacity=0.9] (B.north west) -- (B.north east) -- (B.south east) -- (B.south west) -- (B.north west) -- cycle;
    }
    \end{tikzpicture}
}

% Introduce a new counter for counting the nodes needed for circling
\newcounter{nodecount}
% Command for making a new node and naming it according to the nodecount counter
\newcommand\tabnode[1]{\addtocounter{nodecount}{1} \tikz \node (\arabic{nodecount}) {#1};}


\DeclareMathOperator{\modrelu}{ModReLU}
\DeclareMathOperator{\crelu}{CReLU}
\DeclareMathOperator{\relu}{ReLU}
\DeclareMathOperator{\re}{Re}
\DeclareMathOperator{\im}{Im}
% \DeclareMathOperator{\modrelu}{\text{ModReLU}}
% \DeclareMathOperator{\crelu}{\text{CReLU}}
% \DeclareMathOperator{\re}{\text{Re}}
% \DeclareMathOperator{\im}{\text{Im}}
\DeclareMathOperator*{\argmin}{argmin}
\DeclareMathOperator*{\argmax}{argmax}
\DeclareMathOperator*{\supp}{supp}
\DeclareMathOperator*{\sign}{sgn}
\DeclareMathOperator*{\vari}{var}
\DeclareMathOperator*{\cov}{cov}
\DeclareMathOperator*{\E}{\mathbb{E}}
\DeclareMathOperator*{\prob}{Pr}
\DeclareMathOperator*{\proj}{\mathcal{P}_K}
\DeclareMathOperator*{\support}{\mathcal{S}_K}
\DeclareMathOperator*{\sparse}{\mathcal{H}_K}
\DeclareMathOperator{\bigo}{{\mathcal{O}}}
\DeclareMathOperator{\littleo}{{\scriptstyle\mathcal{O}}}
\DeclareMathOperator{\sublittleo}{{\scriptscriptstyle\mathcal{O}}}
\DeclareMathOperator*{\polylog}{polylog}
\let\originalleft\left
\let\originalright\right
\renewcommand{\left}{\mathopen{}\mathclose\bgroup\originalleft}
\renewcommand{\right}{\aftergroup\egroup\originalright}

\newcommand{\bw}{{\bm w}}
\newcommand{\bx}{{\bm x}}
\newcommand{\be}{{\bm e}}
\newcommand{\by}{{\bm y}}
\newcommand{\ba}{{\bm a}}
\newcommand{\bs}{{\bm s}}
\newcommand{\bW}{{\bm W}}
\newcommand{\bz}{{\bm z}}
\newcommand{\bb}{{\bm b}}
\newcommand{\bl}{{\bm l}}
\newcommand{\bp}{{\bm p}}
\newcommand{\bpsi}{{\bm \psi}}
\newcommand{\bweq}{{\bw_{\mathrm{eq}}}}
\newcommand{\bweqi}{{\bw_{\mathrm{eq}}^{\left\{i\right\}}}}
\newcommand{\bweqt}{{\bw_{\mathrm{eq}}^{\left\{t\right\}}}}
\newcommand{\bweqj}{{\bw_{\mathrm{eq}}^{\left\{j\right\}}}}
\newcommand{\bweqm}{{\bw_{\mathrm{eq}}^{\left\{m\right\}}}}
\newcommand{\bweqk}{{\bw_{\mathrm{eq}}^{\left\{k\right\}}}}
\newcommand{\bweql}{{\bw_{\mathrm{eq}}^{\left\{l\right\}}}}
\newcommand{\bweqistar}{{\bw_{\mathrm{eq}}^{\left\{{i^*}\right\}}}}
\newcommand{\beq}{b_{\mathrm{eq}}}
\newcommand{\beqi}{{b_{\mathrm{eq}}^{\left\{i\right\}}}}
\newcommand{\beqt}{{b_{\mathrm{eq}}^{\left\{t\right\}}}}
\newcommand{\beql}{{b_{\mathrm{eq}}^{\left\{l\right\}}}}

\newcommand{\overbar}[1]{\mkern 1.5mu\overline{\mkern-1.5mu#1\mkern-1.5mu}\mkern 1.5mu}
\newcommand{\abar}{{\overbar{a}}}
\newcommand{\sbar}{{\overbar{s}}}
\newcommand{\ybar}{{\overbar{y}}}
\newcommand{\xbar}{{\overbar{x}}}
\newcommand{\bxbar}{\bm{{\overbar{x}}}}
\newcommand{\bebar}{\bm{\overbar{e}}}
\newcommand{\bsbar}{{\bm{\overbar{s}}}}
\newcommand{\bweqbar}{{\overbar{\bw}_{\mathrm{eq}}}}
\newcommand{\beqbar}{{\overbar{b}_{\mathrm{eq}}}}

\newcommand{\pdfnote}[1]{\marginnote{\pdfcomment[icon=note]{#1}}}

\newcommand{\xhat}{\hat{x}}
\newcommand{\bxhat}{\bm{\hat{x}}}
\newcommand{\behat}{\bm{\hat{e}}}
\newcommand{\bhat}{{\hat{b}}}
\newcommand{\bwhat}{{\bm{\hat{w}}}}

\newcommand{\bwtilde}{{\bm{\tilde{w}}}}

\newcommand{\subN}{{\mathchoice{}{}{\scriptscriptstyle}{}N}}
\newcommand{\subM}{{\mathchoice{}{}{\scriptscriptstyle}{}M}}
\newcommand{\subzero}{{\mathchoice{}{}{\scriptscriptstyle}{}0}}

\DeclarePairedDelimiterX{\infdivx}[2]{(}{)}{%
  #1\;\delimsize\|\;#2%
}
\newcommand{\infdiv}{D\infdivx}


\usepackage{listings}
\lstdefinestyle{python}{
  belowcaptionskip=1\baselineskip,
  breaklines=true,
  frame=shadowbox,
  rulesepcolor=\color{gray},
  xleftmargin=\parindent,
  language=Python,
  showstringspaces=false,
  basicstyle=\footnotesize\ttfamily,
  keywordstyle=\bfseries\color{deepblue},
  moredelim=**[s][\color{blue}]{'''}{'''},
  commentstyle=\itshape\color{magenta},
  identifierstyle=\color{black},
  stringstyle=\color{red}
}
\lstdefinestyle{output}{
  belowcaptionskip=1\baselineskip,
  breaklines=true,
  frame=L,
  basicstyle=\footnotesize\ttfamily,
  xleftmargin=\parindent
}

\urlstyle{rm}
\newcommand\blfootnote[1]{%
  \begingroup
  \renewcommand\thefootnote{}\footnote{#1}%
  \addtocounter{footnote}{-1}%
  \endgroup
}
